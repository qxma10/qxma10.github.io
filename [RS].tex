\documentclass[8pt]{amsart}
\usepackage[letterpaper, margin=1.3in]{geometry}
\usepackage{graphicx}
\usepackage{amsmath,amssymb,amsthm,mathtools}
\usepackage[all]{xy}
\usepackage[utf8]{inputenc}
\usepackage[hidelinks]{hyperref}
\usepackage{enumitem}

\newtheorem{theorem}{Theorem}[section]
\newtheorem{corollary}[theorem]{Corollary}
\newtheorem*{conjecture}{Conjecture}
\newtheorem{remark}{Remark}[section]
\newtheorem{definition}{Definition}[section]
\newtheorem{lemma}[theorem]{Lemma}
\newtheorem{proposition}[theorem]{Proposition}
\newtheorem{question}[theorem]{Question}
\newtheorem{example}[theorem]{Example}
\newtheorem*{theorem*}{Theorem}
\newtheorem*{question*}{Question}
\setcounter{tocdepth}{2}
\begin{document}

\title{Research Statement}

\maketitle

\tableofcontents
%\input{tex/sec0}
\section{Motivations}
\subsection{Finiteness of Brauer group}
Let $X$ proper scheme over $\mathbb{Z}$. Artin posed the beautiful conjecture: $$\#|\mathrm{Br}(X)|<\infty.$$
In case $X$ is a surface over finite field $\mathbb{F}_q$. Taking Galois cohomology of Kummer sequence on $X$, we get
$$0\to\mathrm{NS}(X)/l^m\to\mathrm{H}^2(X,\mu_{l^m})\to\mathrm{Br}(X)[l^m]\to 0.$$ Taking limit, we get $$0\to\mathrm{NS}(X)\otimes\mathbb{Z}_l\to\mathrm{H}^2(\overline{X},\mathbb{Z}_l(1))^{\mathrm{Gal}_{\mathbb{F}_q}}\to T_l\mathrm{Br}(X)\to 0$$ thus Artin's conjecture is equivalent to Tate conjecture for divisors.

If $X$ admits a fibration $f$ onto a curve $C$, with smooth generic fiber $X_\eta$, by the Leray spectral sequence, one can show: $$\mathrm{Br}(X)\cong\mathrm{Sha}(\mathrm{Jac}(X_\eta)).$$ Thus Artin's conjecture implies the BSD conjecture for the Jacobian $\mathrm{Jac}_{X_\eta}$ (because the BSD conjecture is equivalent to finiteness of Sha {\footnote{\label{explainsha}Recall: Given a group scheme $A$ over a global field $K$, its Tate-Shafarevich group $\mathrm{Sha}(A)$ is defined to be the kernel of the natural restriction map: $$\mathrm{H}^1(K,A)\to\prod_{v\in\Sigma_K}\mathrm{H}^1(K_v,A_v),$$
where $v$ runs through all the places of the global field $K$. In other words, $\mathrm{Sha}(A)$ classifies $A$-torsors over $K$, which are locally trivial (with respect to completion).}}
).

The Tate conjecture has been proved for many special cases: abelian varieties, $K3$ surface, product of curves and abelian varieties, product of Shimura curves and Picard modular surfaces, etc\footnote{\href{http://www.jmilne.org/math/articles/2007e.pdf}{\texttt{http://www.jmilne.org/math/articles/2007e.pdf}}}. We do not attempt to approach the conjectures directly. Rather we want to study some similar questions in a more manageable and geometrical setting, hoping to shed some light on the conjectures. Note that the torsors of Jacobian is an essential part of the Brauer group, we hope to understand them explicitly.

$\bullet$ More precisely, let $C/k$ be the universal genus $g$ curve, where $k$ is the function field of $\mathcal{M}_g$, we want to answer the following question: \begin{question}\label{1}Is it true that $$\mathrm{H}^1(k,\mathrm{Jac}_{C})=\langle\mathrm{Jac}^1_{C}\rangle?$$

Or, does every $\mathrm{Jac}_C$ torsor has the form $\mathrm{Jac}^d_C$ for some $d\in\mathbb{Z}$?
\end{question}

The problem may be easier to deal with because: (1) Moduli interpretation enables us to approach the problem via degeneration. (2) When the Jacobian degenerate to a torus, the group $\mathrm{H}^1(k,\mathrm{Jac}_C)$ is known to be finite. (3) There are natural Galois covers of $\mathcal{M}_g$, given by adding level structures. One can study the problem over natural base changes and try to descend.

$\bullet$ The Question \ref{1} is also interesting in its own right. The Picard group of the universal (compactified) Jacobian is computed in \cite{Melo}, the next interesting thing is to calculate its Brauer group.

A first step is to understand the algebraic part of the Brauer group, which is essentially understanding $\mathrm{H}^1(k,\mathrm{Jac}_C)$. A second step is to understand $\mathrm{Br}(\mathcal{M}_{g,n})$. There are some calculation for $\mathcal{M}_{1,1}$, see \cite{M11}, but the general answer is unknown (However, the general answer maybe boring, because it is not motivated by any fun applications.).

There are calculations for Brauer Group of moduli space of rank $r\geq2$ vector bundles over curves, see \cite{Biswas2}, \cite{Biswas}. There, the Brauer group is shown to be a cyclic group generated by the class of the projectivized tautological bundle
\footnote{For $r=1$, or the case Question \ref{1} concerned, the paper arXiv:1805.05369 claimed the same is true, but their key Proposition 3.5 is wrong. However the wrong proof motivates a possible approach, see \ref{bsv}.}.

\subsection{Cyclicity Problem}\label{2} (This thing I think is not doable, because there seems to be no good characterization of ramified cyclic coverings, $k^*/k^{*n}$ is too big and has no structures in general.)
Let $k$ be a field, let $A$ be a $k$-central simple algebra. We know by Wedderburn's theorem that $A\otimes_k\overline{k}\cong\mathrm{M}_n(\overline{k})$. The number $n=\sqrt{\mathrm{dim}_k(A)}$ is called the degree of $A$.
We say $A$ is a cyclic algebra, if there exists a Galois extension $L/k$ of degree $n$ that splits $A$ (i.e. $A\otimes_kL\cong M_n(L)$).

Here is a big open problem for division algebraists: Is every central simple algebra of prime degree\footnote{Central simple algebras of prime degree are automatically division algebras.} cyclic?

People care about this because: Division algebras of prime degree can be thought as building blocks of central simple algebras, and cyclic algebras have very simple structure $k\langle x,y|x^n=a,y^n=b,xy=\zeta_nyx\rangle$.

It is shown that division algebras of degree $2$,$3$ are always cyclic, and there exist examples of non-cyclic algebra of degree $4$. It is unknown if there exists a non-cyclic division algebra of degree $5$.

The cyclicity problem reflects arithmetic properties of the fields. Over global fields, using class field theory, one can shows every central simple algebra is cyclic. Saltman proved division algebra over $\mathbb{Q}_p(t)$ of degree $p$ are cyclic \cite{Salt}.


People tried various ways to approach the problem.

$\bullet$ Here is an idea due to Saltman: Note that a Brauer class is split by its Brauer-Severi variety. The cyclic problem will be solved, if we can always find a closed point in a Brauer-Severi variety of dimension $p-1$, whose geometric points form a $\mathbb{Z}/p\mathbb{Z}$-orbit.

Assume we can do so, then joining the closed points by $\mathbb{P}^1$ we obtain a loop of rational curves. We can think of this as degeneration of a genus $1$ curve. Then we can find a genus $1$ curve in Brauer-Severi variety, after base change to $k((t))$.
A fun result: For $p\leq5$, every Brauer-Severi scheme of dimension $p-1$ contain a genus $1$ curve, see \cite{dJ}.

The existence of genus one curve in Brauer-Severi varieties with large index is also interesting in its own right, for example, this implies the existence of elliptic curves with arbitrarily large index\footnote{a nontrivial result \cite{Stein}, proved using Kolyvagin's method, a method also used in proving certain inequality of BSD.}. General degenerate curves in Brauer-Severi varieties are also considered in \cite{KollarB}. We may ask if there always exist genus $g$ curves in every Brauer-Severi variety of dimension $g-2$. (One idea is to cut out curves by sections of vector bundles, but the genus grows large very quickly.)

$\bullet$ Recall there is a Brauer obstruction class to the existence of a universal line bundle on $C\times\mathrm{Pic}^0_{C/k}$, denote the class by $\alpha\in\mathrm{Br}(\mathrm{Pic}^0_{C/k})$. De Jong thought $\alpha$ might be non-cyclic, there might be chance to study $\alpha$ via degeneration and show the local cyclicity information do not patch up, giving non-cyclicity. Two obstructions to overcome: One is the difficulty to measure (or attach invariants to) cyclicity. Secondly, for the loci where we have manageable degeneration, the codimension are too high. Cohomological purity \cite{Purity2} for Brauer group erases all information above codimension two. Thus if we were to proceed, we need to study Azumaya algebras that represent the classes, because they do not have purity, see \cite{Ant}. And it is in general very interesting to know how Azumaya algebras give feedback on high codimensional information. One concrete example to understand is:
\begin{question}\label{2}Does the class $\alpha$ contain an Azumaya order?
\end{question}
Let $\pi\colon C\times\mathrm{Pic}^0_{C/k}\to\mathrm{Pic}^0_{C/k}$ be the projection, let $\mathcal{L}$ be the tautological line bundle on $C\times\mathrm{Pic}^0_{C/k}$. The natural Azumaya algebra $\mathcal{E}nd(\pi_*(\mathcal{L}\otimes\omega_C))$ does not extend to a locally free sheaf over the identity section. But things become mysterious if we are allowed to replace the algebra by Morita equivalent ones\footnote{In general, when we study the Brauer group in terms of Azumaya algebras, Morita equivalence is always a key difficulty to overcome, for example, the proof $\mathrm{Br}=\mathrm{Br}'$, see \href{http://www.math.columbia.edu/~dejong/papers/2-gabber.pdf}{\texttt{http://www.math.columbia.edu/$\sim$dejong/papers/2-gabber.pdf}}.}.

$\bullet$ Max Lieblich observed the Brauer group of supersingular $K3$ surfaces are uncountable, and wondered if that could be used to prove cyclicity.

\subsection{Geometry of liftings}
There are lots of unanswered questions around lifting varieties.

$\bullet$ Let $R$ be a discrete valuation ring. Let $X/R$ be a smooth projective variety, let $X_\eta$ and $X_s$ be the generic fiber and special fiber. Here is a very natural question:
\begin{question} Do $X_\eta$ and $X_s$ have same Kodaira dimension?
\end{question}
Siu shown the plurigenera $\mathrm{H}^0(X,\Omega_X^k)$ is constant if $R$ is of equal characteristic zero. There are examples of jumping Hodge numbers in characteristic $(0,p)$ and $(p,p)$. We do not know if the jump could be large enough to yield different Kodaira dimensions. For surfaces, we know the Kodaira dimension is always constant in families, see \cite{Hacon}.

$\bullet$ Deligne and Illuise gave an algebraic proof of the Kodaira vanishing theorem \cite{DI}. There the key result is: Given a smooth projective variety $X$ over a characteristic $p$ perfect field $k$, suppose $X$ admits a lift to $W_2(k)$. Let $X\to X'$ be the relative Frobenius map, then the truncated de-Rham complex $\tau_{<p}F_*\Omega^\bullet_{X/k}$ is decomposable in $D^b(X')$. The proof breaks down in degree $p$ because $\wedge^p$ is not a summand of $\otimes^p$ in characteristic $p$. Deligne and Illusie asked:
\begin{question}Is the decomposability true without truncation?
\end{question}
Since decomposability implies Kodaira vanishing and Hodge-de Rham degeneracy, it suffices to find liftable varieties without these properties. (We may also try to construct examples by iteratedly taking fiber bundles and ramified coverings, check liftability in each step, and see if we are lucky.)

The difficulty is produce varieties which are $W_2$-liftable.
Note that finite group schemes over $W_2$ are classified, there are concrete but tedious ways to test liftability. Let $G$ be a finite group scheme (in this case, the classification stack $BG$ is proper smooth).
If we find counterexample in cohomologies $BG$, then it is routine to approximate them by varieties. The cohomology of $BG$ can be calculated by hypercovering spectral sequence \cite{Tot}, the problem that $\wedge^p$ is not summand of $\otimes^p$ also occurs here.

Ogus remarked that the nonabelian Hodge theory: equivalence of modules with flat connection $\mathrm{Mic}_{\leq p-1}(X)$ and nilpotent Higgs bundles $\mathrm{Higgs}_{\leq p-1}(X')$ is a categorification of the decomposability \cite{Volo}. One may wonder if one can show non-decomposability by the unequivalence of categories? (Ok, I think this is also not quite workable.) For curves, the is full decomposability and full equivalence, see \cite{Xin}.

$\bullet$ Motivation also come from Bhargav's work, where the topology of generic fiber is controlled by the algebraic structure of the special fiber \cite{BMS}: $$h^i_{dR}(X_s)\geq h^i_{\acute{e}t}(X_{\overline{\eta}},\mathbb{F}_p).$$
One could ask how this inequality could fail for $i=1$? For example, if $X_{\overline{\eta}}$ is simply connected, can the special fiber have a non-exact closed $1$-form? Can there be arbitrarily many independent of them? Is there a nice description of the difference (given many nice interpretations of $\mathrm{H}^1$)?

$\bullet$ For a pool of interesting lifting questions around abelian varieties, see a survey by Oort \footnote{\href{https://www.birs.ca/workshops/2015/15w5035/files/OortLiftingProblems.pdf}{\texttt{https://www.birs.ca/workshops/2015/15w5035/files/OortLiftingProblems.pdf}}}.

\subsection{Application of Langlands}
(This is not currently workable, as I do not have enough working knowledge.) It is really interesting to see how the results in Langlands program (beyond class field theory) can be applied to concrete problems in number theory or algebraic geometry.

$\bullet$ For geometric langlands, one application is that the count of certain $\overline{\mathbb{Q}}_l$ local systems on curves is independent of $l$, because they can be viewed on the automorphic side, which has nothing to do with $l$ \cite{Dri}. There are some other examples, see Kedlaya's webpage\footnote{\href{https://kskedlaya.org/galc.shtml}{\texttt{https://kskedlaya.org/galc.shtml}}}. It is interesting to find some more concrete and less artificial applications.

$\bullet$ It is interesting to see how modularity\footnote{According to a quote attributed to Eichler, the five elementary mathematical operations are addition, subtraction, multiplication, division and modular forms.} can be applied to geometry. Modular curves\footnote{They can be constructed as moduli of elliptic curves with various level structures.} form a very special class of objects. We say a $C$ curve satisfy modularity if its $\mathrm{Jac}_C$ is dominated by Jacobian of a modular curve.

Modular curves appeared in the construction of Hilbert class field of imaginary quadratic fields. Modularity of elliptic curves with semistable reduction (we do not know how to construct the map geometrically) is used to bound ramification of Galois representation, leading to the Fermat's last theorem. The Tate-Shafarevich group of Jacobian of certain modular curves are proved to be finite. Modular curves have good reduction theory (higher dimensional Shimura varieties do not yet have nice reduction theory). It is interesting to understand how modular curves work in many different settings, and apply them to more situations.





%\input{tex/sec1}

\section{Concrete Problems}
\subsection{Workable problems}
\begin{enumerate}
\item
 Let $A$ be a group scheme over $k$. Let's denote $\mathrm{H}^1(k,A)$ by $\mathrm{WC}(A)$ (the Weil-Chatelet group). In order to approach Question \ref{1}, we try to calculate the WC group of certain ``universal'' curves.

    The simplest case is genus $1$. There does not exist a universal elliptic curve, but there is a universal uniformization of elliptic curves with semistable reduction: the Tate curve. Let $K=\mathbb{C}((t))$. The Tate curve $E_t$ can be constructed as quotient $$0\to t^\mathbb{Z}\to K^*\to E_t\to 0.$$ The quotient is $\widehat{\mathbb{Z}}=\mathrm{Gal}(\overline{\mathbb{C}((t))}/\mathbb{C}((t)))$-invariant. The long exact sequence implies that $WC(E_t)=\mathbb{Q}/\mathbb{Z}$, which seems to be generated by one element.

    For higher genus, for curves with semistable reduction, fix a degeneration type, there exist similar universal uniformization by Mumford-Schottky. It is known that their Jacobians also have nice uniformization, see \cite[VI]{Scho}. One may calculate WC group of the Jacobian of Mumford-Schottky curves and see what happens.
\item Calculate the $\mathrm{WC}$ group of Jacobian torus of universal degenerate curves. Since they are known to be finite, can we determine all the elements? Is it a 2-torsion group generated by torsors $[\mathrm{Pic}^{1,0,\dots,0}]$, etc? Then see what the kind of restriction do they put on the generic fiber.
\item We know the Brauer group of a curve of algebraically closed field is trivial, by Tsen's theorem. It is unclear what is the Brauer group of the universal curve, or universal degenerate curves. For smooth curve, this is the same as Question \ref{1}.
\item The Fano surface of a cubic threefold has many analogies with curve. This is a general type surface, its Jacobian has dimension $5$. The Jacobian admits autoduality, the theta divisor has a unique normal singularity whose tangent cone recovers the ambient cubic threefold. We want to study the finiteness of Brauer group of the universal Fano surface $S$. The Fano surface embeds into its Jacobian, we wonder if the tautological obstruction class restricts to a generator of $\mathrm{Br}(S)$. \footnote{The Hodge conjecture is known to be true for the Fano surface of cubic threefolds, based on the Hodge conjecture of low degree hypersurfaces, mabye the Tate conjecture for such Fano surfaces is also proved.}
\item Brauer group moduli of rank $r$ bundles has been calculated for $r\geq2$. Check if this have any implication on the WC group of Jacobian. Examine where the proof fails for rank $1$\footnote{Is purity the only problem?}.
\end{enumerate}


\subsection{Ideas to be developed, A}
\begin{enumerate}

\item Brauer group of Brauer-Severi varieties. {\begin{question}\label{bsv}Let $k$ be a field, let $P/k$ be a Brauer-Severi scheme representing a class $\alpha$. We know $\alpha$ restricts to zero on $P$. Does the pullback induce an isomorphism: $$\mathrm{Br}(P)\cong\mathrm{Br}(k)/\langle\alpha\rangle?$$\end{question} Note that suitable symmetric product of curves $C$ are Brauer-Severi schemes over $\mathrm{Jac}_C$. If we understand this situation well, we may recover the Brauer group of $\mathrm{Pic}^0_C$. See here \footnote{\href{https://pbelmans.ncag.info/blog/2016/04/20/the-brauer-group-of-brauer-severi-varieties/}{\texttt{https://pbelmans.ncag.info/blog/2016/04/20/the-brauer-group-of-brauer-severi-varieties/}}} for some discussion.}
\item Study some other special cases of Brauer group is known to be finite, can we explicitly describe all the elements?
\item Behavior of Brauer class over singularities. This is another direction in understanding how to extract high codimensional information from Azumaya algebras, motivated by section \ref{2}. A concrete example to understand: We can show the Brauer class over the Picard scheme of curves restricts to zero at the generic point of the theta divisor, but we do not know if it restrict to zero on the whole theta divisor (which carries singularities).
\item ``Nonflat Brauer-Severi schemes''. These are analogs of projectiviation of coherent sheaves. We want to study its purity (true in char $0$) and behavior at singularities. The goal is 2.4.(\ref{c}).
\end{enumerate}

\subsection{Ideas to be developed, B}
\begin{enumerate}
\item Franchetta conjecture asserts that the only rational points of the universal Picard $\mathrm{Pic}^0_{C/k}$ is the identity section. What are the degree $2$ closed points in $\mathrm{Pic}^0_{C/k}$?
\item This question and the next rise in the study of stable birationality problems of cubic hypersurfaces. Let $X$ be a cubic surface, we know geometrically $X$ is the blowup of $\mathbb{P}^2$ in six points. We ask: Is $\mathrm{Sym}^3(X)$ rational? (Note $\mathrm{Sym}^3(X)$ is geometrically rational and has a rational subset parameterized by intersection with lines in $\mathbb{P}^3$. When $X$ is the universal cubic surface, does there exist non-collinear degree 3 points?)
\item Let $m,n$ be even integers, $m<n+3$. Let $P$ be a degree $m$ closed point in $\mathbb{P}^n$. Consider the moduli space $F_{P,n}$ of rational normal curves passing through $P$. Is $F_{P,n}$ rational? In case $m=4,n=2$, the space parameterize conics passing through $4$ points, we know $F_{P,n}$ is rational, as conics are complete intersection. If $n\geq4$, things become unclear.
\item The question mentioned before in Section 1.3: can we find a smooth variety over mixed char DVR, whose generic fiber is simply connected, but special fiber has non-exact $1$-forms?
\item The question mentioned before in Section 1.3: calculating cohomology of classification stack of more complicated liftable group schemes, e.g. kernel of isogeny between abelian varieties, non-reduced stabilizer of Lie group action of homogeneous spaces. See if Hodge to de Rham degeneracy can fail.


\end{enumerate}
\subsection{Wild thoughts}
\begin{enumerate}
\item There is a version\footnote{\href{https://arxiv.org/abs/1604.02939}{\texttt{https://arxiv.org/abs/1604.02939}}} of Franchetta conjecture for zero cycles on universal polarized $K3$ surfaces. The degeneration of $K3$ surfaces are well-understood. There are also theory degeneration of pairs. Can we use degeneration techniques to approach the problem? (In the curve case, we can reduced our problem to triviality of certain torsor. But the chow group of zero cycles is quite different.)
\item Let $p$ be a prime, let $J_{X_0(N)}$ be the Jacobian of modular curve $X_0(N)$, in some cases, it is known that $\mathrm{Sha}(J_{X_0(N)})$ is finite. Can we explicitly describe its elements? When $X_0(N)$ has genus $1$, see \cite{Mazur}. How about some higher genus cases?
\item\label{c} Some idea for non-cyclicity: note that cyclicity is stable under specialization. Although we do not have examples of non-cyclic algebra of prime degree, we have non-cyclic algebra of $p^2$ degree. Is it possible to fit a non-cyclic degree $p^2$ algebra into an Azumaya algebra of degree $p$?
\item Understand Brauer group over stacks, which of them descend to the coarse moduli?
\item By Bondal-Orlov reconstruction theorem, derived categories of a general type variety uniquely recovers the variety. We can read off the Picard group using invertible elements. Can we recover the Brauer group? How about some special cases, product of curves, K3 surfaces?
\item Given a hyperelliptic curve over finite field. When is it Jacobian supersingular?
\item Does uncountability of Brauer group of supersingular K3 surface(over algebraically closed field) imply cyclicity?
\end{enumerate}


\bibliographystyle{alpha}
\bibliography{references}

\end{document} 